\chapter{Conclusion}
In this Thesis we developed a code for solving the time-dependent Schr\"odinger equation of a wave function in a two-dimensional space. The code implements a solver for quantum systems described by Hamiltonians with a static external potential and a self-interacting term. Furthermore, the code includes a solver for the imaginary time evolution, to approximate the ground state of the system.

This result has been obtained by extending  a recent work of Wittek and Cucchietti~\citep{Wittek20131165}, in which they developed a solver for a free quantum particle that scales to massively parallel computing clusters. The code implements the algorithm on four different kernels: two cache optimized kernels for central processing unit (CPU) -- one is further optimized to use the SSE instructions --, a kernel for general-purpose graphics processing unit (GPU) and a hybrid kernel that use both CPUs and GPUs. The algorithm is based on the second order of the Trotter--Suzuki approximation, which provides an accurate approximation of the evolution operator. Moreover, the approximation leads to an algorithm easy to parallelise that results in an efficient distribution of the workload across the nodes of a cluster. Indeed, the CPU kernels show a linear scaling of the throughput, so that doubling the number of cores, involved in the calculation, results in a halved time of execution. This can lead to very fast simulations on big clusters. On the other hand, GPU and hybrid kernels obtain better performances on a smaller scaled system, which makes them preferable on a single workstations.

We proved the accuracy of our code reproducing the results obtained by an experiment carried out at the National Institute of Standard and Technology (NIST)~\citep{DSF00}. We approximated the ground state of a Bose--Einstein condensate (BEC) of sodium atoms, confined in a magnetic trap, and simulated the evolution of the state that underwent on a phase imprinting. We were able to see the generation of a soliton and other excitations in agreement with the experimental observations.

We achieved this by mean of the approximation described in Ref.~\citep{PietroMassignan}. This approximation let us reduce the three dimensional Gross--Pitaevskii equation, which provides a model of the BEC, to a two dimensional model solvable by our code.

The software comes with the General Public License and it can be redistributed under the terms of this license. We developed an application programming interface (API) that exposes the function that performs the evolution. Furthermore, the CPU and SSE kernels are accessible from Python and MATLAB\footnote{For more information about how to get the code visit the web site https://github.com/peterwittek/trotter-suzuki-mpi.}.

A clear direction for future extensions of the code is the implementation of the nonlinear Schr\"odinger equation to all the kernels -- at the moment it is only supported on the CPU kernel. These would be useful in many fields of physics, for instance in ultracold atom physics, in which the Gross--Pitaevskii equation plays a major role in describing the systems. In this case, the phenomenology to be studied include vortexes, spin-orbit coupling~\citep{GS13} and quantum thermalization. Other applications may regards the simulations of cloak system and quantum holography.

The method used to rescale a three dimensional problem to  a two dimensional one may not be always applicable. This motivates the extension to three dimensions. The variety of possible decomposition strategies in this case is large, and a flexible implementation would be very useful to test out the performance of the different choices.