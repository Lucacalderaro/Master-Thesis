\chapter{Soliton propagation in Bose-Einstein condensate}


%%%%%%%%%%%%%%%%%%%%%%%%%%%%%%%%%%%%%%%%%%%%%%%%%%%%%%%%%%%%%%%%%%%%%%%%%%%%%%%%%%%%
\section{Gross-Pitaevskii equation}
The dynamic of a Bose-Einstein condensate (BEC) may be described by a non linear Schr\"odinger equation, namely the Gross-Pitaevskii equation. Suppose we have a $N$-particle system, comprised of bosons interacting with each other. Using the Hartree approximation, we write the many-body wave function as
\begin{equation}
\psi(\textbf{r}_1, \textbf{r}_2,\ldots, \textbf{r}_N) = \prod_{i=1}^N \phi(\textbf{r}_i),
\end{equation}
where the single-particle wave function $\phi(\textbf{r}_i)$ is normalized in the usual way,
\begin{equation}
\int \mathrm{d}\textbf{r} \, |\psi(\textbf{r})|^2 = 1.
\end{equation} 
For low energy particles the interaction can be described by an effective potential %(PS6.1)
that in coordinate space corresponds to a contact interaction $U_0 \delta(\textbf{r} - \textbf{r}')$, where $U_0 = 4 \pi \hbar^2 a / m$ and $a$ is the scattering length of the interaction. Considering an external potential $V(\textbf{r}_i)$, the Hamiltonian may be written as
\begin{equation}
H = \sum_{i=1}^N \left[ \frac{\textbf{p}_i^2}{2 m} + V(\textbf{r}_i) \right] + U_0 \sum_{i<j} \delta(\textbf{r}_i - \textbf{r}_j).
\end{equation}
The energy of the state is given by
\begin{equation}
E =  N \int \mathrm{d} \textbf{r} \, \left[ \frac{\hbar^2}{2m} |\nabla \psi(\textbf{r}) |^2 + V(\textbf{r}) |\psi(\textbf{r})|^2 + \frac{(N-1)}{2} U_0 |\psi(\textbf{r})|^4 \right].
\end{equation} 
It is convenient to introduce the concept of the wave function of the condensed state,
\begin{equation}
\psi(\textbf{r}) = N^{1/2} \phi(\textbf{r}),
\end{equation}
so that the normalization condition is
\begin{equation}
\int \mathrm{d}\textbf{r} \, |\psi(\textbf{r})|^2 = N.
\end{equation}
Moreover, we suppose that $N \gg 1$ so that the energy can be rewritten as,
\begin{equation}
E = \int \mathrm{d} \textbf{r} \, \varepsilon  = \int \mathrm{d} \textbf{r} \, \left[ \frac{\hbar^2}{2m} |\nabla \psi(\textbf{r}) |^2 + V(\textbf{r}) |\psi(\textbf{r})|^2 + \frac{1}{2} U_0 |\psi(\textbf{r})|^4 \right].
\end{equation}
The equation of motion may be derived from the principle of least action 
\begin{equation}
\delta \int_{t1}^{t2} \mathrm{d} t \, L = 0 \label{eq:least-principle}
\end{equation}
where the Lagrangian $L$ is given by
\begin{equation}
L = \int \mathrm{d} \textbf{r} \, \left[ \frac{\imath \hbar}{2} \left( \psi^\ast \frac{\partial \psi}{\partial t} - \psi \frac{\partial \psi^\ast}{\partial t} \right) - \varepsilon \right].
\end{equation}
Requiring that the variation of the independent variables $\psi(\textbf{r},t)$ and $\psi(\textbf{r},t)$ vanish at $t=t_1$ and $t=t_2$, and on any spatial boundaries, from Eq.~\eqref{eq:least-principle} one finds the Gross-Pitaevskii equation:
\begin{equation}
\imath \hbar \frac{\partial \psi(\textbf{r}, t)}{\partial t} = - \frac{\hbar^2}{2m} \nabla^2 \psi(\textbf{r}, t) + V(\textbf{r}) \psi(\textbf{r}, t) + U_0 |\psi(\textbf{r}, t)|^2 \psi(\textbf{r}, t).
\end{equation}

The ground-state of the system may be found numerically using the imaginary time evolution. However, an accurate expression is obtained using the Thomas-Fermi approximation. Suppose that the particles have low kinetic energy, then the ground-state is approximated by the solution of the time independent Gross-Pitaevskii equation deprived of the kinetic term
\begin{equation}
\left[ V(\textbf{r}) + U_0 |\psi(\textbf{r})|^2 \right] \psi(\textbf{r}) = \mu \psi(\textbf{r}),
\end{equation}
where $\mu$ is the chemical potential. This give the solution
\begin{equation}
n(\textbf{r}) = |\psi(\textbf{r})|^2 = \frac{\mu - V(\textbf{r})}{U_0},
\end{equation}
for $\textbf{r}$ such that $V(\textbf{r}) > \mu$. 
For a BEC trapped in a harmonic potential of the form
\begin{equation}
V(x,y) = \frac{m}{2} (\omega_1^2 x^2 + \omega_2^2 y^2)
\end{equation}
the extension of the condensate wave function in the two directions is given by the two semi-axes
\begin{equation}
R_i^2 = \frac{2\mu}{m\omega_i^2}, \, i = 1,2
\end{equation}
and the particles density has the form of an upside down parabola. The Tomas-Fermi approximation gives an accurate description of the bulk properties of the system, but it fails near the edge of the cloud.
%ref PS6.3
