\chapter{Decomposition of Unitary Evolution}
When it comes to solving differential equations that determine the behaviour of a physical system, an accurate approximation is often needed to find an explicit solution. As we have seen in the previous chapter, a good approach is to use the Trotter--Suzuki approximation. 

In this chapter we are interested in calculating the evolution operator $U(t) = \exp \left( - \frac{\imath t}{\hbar} H \right)$, the solution of the time--dependent Schr\"odinger equation, using the second order Trotter--Suzuki approximation. The Trotter--Suzuki approximation is based on the splitting of the exponent in a sum of operators. Thus, starting from a exponential that is difficult to calculate, we end up with a product of easy--to--calculate exponentials. In the next sections, we discuss how to decompose the Hamiltonian and the eventual calculation of the exponential.

%%%%%%%%%%%%%%%%%%%%%%%%%%%%%%%%%%%%%%%%%%%%%%%%%%%%%%%%%%%%%%%%%%%%%%%%
\section{Hamiltonian Decomposition}
We consider a single quantum particle in two dimensions in time--independent potential. The Hamiltonian operator of such system is written as follows
\begin{equation} \label{eq:hamiltonian-implementation}
\hat{H} = \frac{\hat{P}_x^2 + \hat{P}_y^2}{2m} + \hat{V},
\end{equation}
where $m$ is the particle mass and $\hat{V}$ is the external potential.

We use the coordinate reppresentation of the operators, so the kinetic term becomes
\begin{subequations}
\begin{align}
\bra{x,y} \frac{1}{2m} \left( P_x^2 + P_y^2 \right) \ket{\psi} = & \int \mathrm{d}x' \mathrm{d}y' \bra{x,y} \left( P_x^2 + P_y^2 \right) \ket{x',y'} \braket{x',y'|\psi} \label{eq:tot-momentum1} \\
= & - \frac{\hbar^2}{2m} \left( \nabla_x^2 + \nabla_y^2 \right) \psi(x,y) \label{eq:tot_momentum}
\end{align}
\end{subequations}
In this basis, the exponentiation of the external potential operator is straightforward, since it is diagonal. On the contrary, this is not true for the kinetic operator.

We consider the discretization of the continuum space into a uniform mesh, where $\Delta$ is the distance between any two consecutive points. We use the tuple $(i,j)$ to label the points of the mesh, with $i,j = 1,\ldots , N$, so that $\psi(x,y) \rightarrow \psi_{i,j}$ and $\ket{x,y} \rightarrow \ket{i,j}$. Using the second--order derivative central difference, we have:
\begin{equation} \label{eq:second-order-derivative}
\frac{\partial^2 \psi}{\partial x^2} \bigg|_{i,j} = \frac{\psi(i+1,j) - 2 \psi(i,j) + \psi(i-1,j)}{\Delta^2} + O(\Delta^2),
\end{equation}
Then we can write the Eq.~\eqref{eq:tot_momentum} as
\begin{equation} 
\bra{i,j} \frac{1}{2m} \left( P_x^2 + P_y^2 \right) \ket{\psi} = -\frac{\hbar^2}{2m\Delta^2} \left( \psi_{i+1,j} + \psi_{i,j+1} + \psi_{i-1,j} + \psi_{i,j-1} - 4 \psi_{i,j} \right). \nonumber
\end{equation}
To explicitly determine the matrix elements of the kinetic operator, let us rewrite the previous equation using Kroneker's delta
\begin{align} \label{eq:delta-k-momentum}
\bra{i,j} \frac{1}{2m} \left( P_x^2 + P_y^2 \right) \ket{\psi} = & \sum_{k,l} -\frac{\hbar^2}{2m\Delta^2} [ \left(  \delta_{i+1,k} + \delta_{i-1,k} \right) \delta_{j,l} \, +  \\ 
+ &  \left( \delta_{j+1,l} + \delta_{j-1,l} \right) \delta_{i,k} - 4 \delta_{i,k} \delta_{j,l} ] \psi_{k,l}. \nonumber
\end{align}
Since the discretization of Eq.~\eqref{eq:tot-momentum1} led to the following equation
\begin{subequations}
\begin{align}
\bra{i,j} \frac{1}{2m} \left( P_x^2 + P_y^2 \right) \ket{\psi} = & \frac{1}{2m} \sum_{k,l} \bra{i,j} \left( P_x^2 + P_y^2 \right) \ket{k,l} \braket{k,l|\psi} \\
= & \frac{1}{2m} \sum_{k,l} \bra{i,j} \left( P_x^2 + P_y^2 \right) \ket{k,l} \psi_{k,l} \label{eq:discr-momentum}
\end{align}
\end{subequations}
from the comparison of Eq.~\eqref{eq:delta-k-momentum} and Eq.~\eqref{eq:discr-momentum}, we get
\begin{align} \label{eq:delta-kroneker}
\bra{i,j} \frac{1}{2m} \left( P_x^2 + P_y^2 \right) \ket{k,l} = -\frac{\hbar^2}{2m\Delta^2} & [ \left( \delta_{i+1,k} + \delta_{i-1,k} \right) \delta_{j,l} \, + \\ + & \left( \delta_{j+1,l} + \delta_{j-1,l} \right) \delta_{i,k} - 4 \delta_{i,k} \delta_{j,l} ]. \nonumber
\end{align}

We introduce two operators that will let us split the Hamiltonian into a sum of operators that are easy to exponentiate. We define:
\begin{equation} \label{eq:A-form}
A_{i,k} = \begin{cases} \delta_{i+1,k}, & \mbox{if } k\mbox{ is odd} \\ \delta_{i-1,k}, & \mbox{if } k\mbox{ is even} \end{cases}
\end{equation}
and
\begin{equation} \label{eq:B-form}
B_{j,l} = \begin{cases} \delta_{j-1,l}, & \mbox{if } l\mbox{ is odd} \\ \delta_{j+1,l}, & \mbox{if } l\mbox{ is even} \end{cases}
\end{equation}
Represented as matrices, these operators have the form of block diagonal matrices, namely:
\begin{equation}
A = \begin{pmatrix}
0 & 1 \\
1 & 0 \\
 & & 0 & 1 \\
 & & 1 & 0 \\
 & & & & 0 & 1 \\
 & & & & 1 & 0 \\
 & & & &  & & \ddots \\
\end{pmatrix}
\quad B = \begin{pmatrix}
0 \\
& 0 & 1 \\
& 1 & 0 \\
& & & 0 & 1 \\
& & & 1 & 0 \\
& & & & & 0 & 1 \\
& & & & & 1 & 0 \\
& & & & &  & & \ddots \\
\end{pmatrix}
\end{equation}
Using the new operators we can rewrite Eq.~\eqref{eq:delta-kroneker} as follow
\begin{align}
\bra{i,j} \frac{1}{2m} \left( P_x^2 + P_y^2 \right) \ket{k,l} = -\frac{\hbar^2}{2m\Delta^2}  [ & \left( A_{i,k} + B_{i,k} \right) \delta_{j,l} \, + \\ + & \left( A_{j,l} + B_{j,l} \right) \delta_{i,k} - 4 \delta_{i,k} \delta_{j,l} ] \nonumber
\end{align}
For the brevity of notation, we adopt the operator notation, so that the previous equation becomes
\begin{equation} \label{eq:mom-decomposition}
\frac{1}{2m} \left( \hat{P}_x^2 + \hat{P}_y^2 \right)= -\frac{\hbar^2}{2m\Delta^2} \left[ \hat{A}_x + \hat{B}_x + \hat{A}_y + \hat{B}_y - 4 \hat{I} \right],
\end{equation}
where the label indicates the index to which the operator acts, so that the following commutation rules are satisfied:
\begin{equation}
[\hat{A}_x, \hat{A}_y] = 0 \quad [\hat{A}_x, \hat{B}_y] = 0 \quad [\hat{B}_x, \hat{A}_y] = 0 \quad [\hat{B}_x, \hat{B}_y] = 0.
\end{equation} 
Finally, by mean of Eq.~\eqref{eq:mom-decomposition}, we get the decomposition formula for the Hamiltonian~\eqref{eq:hamiltonian-implementation}:
\begin{equation} \label{eq:hamiltonian-decomposition}
\hat{H} = -\frac{\hbar^2}{2 m \Delta^2} \left[ \hat{A}_x + \hat{B}_x + \hat{A}_y + \hat{B}_y - 4 \hat{I} \right] + \hat{V}.
\end{equation}

%%%%%%%%%%%%%%%%%%%%%%%%%%%%%%%%%%%%%%%%%%%%%%%%%%%%%%%%%%%%%%%%%%%%%%%%%%%%%%%%%%%%%%%%%
\section{Evolution Operator}
In the previous section, we shown how to split the Hamiltonian in the discrete space approximation. Now we explicitly calculate the Trotter--Suzuki decomposition for the evolution operator. Using the Hamiltonian decomposition~\eqref{eq:hamiltonian-decomposition}, the evolution operator $\hat{U}(t) = \exp(-\frac{\imath t}{\hbar} H)$ can be written as follows, in the first Trotter--Suzuki approximation
\begin{align} \label{eq:1approxTS}
\hat{U}_1(t) = \exp\left(-\frac{\imath t}{\hbar}\left(\hat{V} + \frac{2 \hbar^2}{m \Delta^2} \hat{I}\right) \right) & \exp\left(\imath \alpha \hat{A}_x \right) \exp\left(\imath \alpha \hat{B}_x \right) \cdot \\ \cdot & \exp\left(\imath \alpha \hat{A}_y \right) \exp\left(\imath \alpha \hat{B}_y \right) + O(t^2) \nonumber
\end{align}
where we defined $\alpha = \frac{\hbar t}{2m\Delta^2}$. 

Using the equality 
\begin{equation}
\exp(\imath \alpha \sigma) = I \cos(\alpha) + \imath \sigma \sin(\alpha),
\end{equation}
it is straightforward to calculate the exponential of the operators $A$ and $B$, since they are diagonal matrices of the Pauli matrix:
\begin{equation} \label{eq:expA}
\exp\left(\imath \alpha \hat{A}_x \right) = \hat{I}_x \cos(\alpha) + \imath \sin(\alpha) \hat{A}_x
\end{equation}
\begin{equation} \label{eq:expB}
\exp\left(\imath \alpha \hat{B}_x \right) = \hat{I}_x (\cos(\alpha)(1-\delta_{x,0}) + \delta_{x,0}) + \imath \sin(\alpha) \hat{B}_x
\end{equation}
The first exponential in Eq.~\eqref{eq:1approxTS} is also straightforward to calculate. Indeed $\hat{V} \ket{i,j} = V(i,j) \ket{i,j}$ so,
\begin{equation}
\bra{k,l} \exp\left(-\frac{\imath t}{\hbar}\left(\hat{V} + \frac{2 \hbar^2}{m \Delta^2} \hat{I}\right) \right) \ket{i,j} = \delta_{k,i} \delta_{l,j} \exp\left( -\frac{\imath t}{\hbar} \left( V(i,j) + \frac{2 \hbar^2}{m \Delta^2} \right) \right) \nonumber
\end{equation}
The second order approximation of Trotter--Suzuki decomposition is easily calculated using $U_1(t)$, namely
\begin{equation} \label{eq:second-TS-evo-operator}
\hat{U}_2(t) = \hat{U}_1\left( -\frac{t}{2} \right)^\dagger \hat{U}_1\left(\frac{t}{2}\right)
\end{equation}

%%%%%%%%%%%%%%%%%%%%%%%%%%%%%%%%%%%%%%%%%%%%%%%%%%%%%%%%%%%%%%%%%%%%%%%%%%%%%%%%%%%%%%%%%
\section{Evolution Towards the Ground-state}
A reliable and easily--implemented method of approximating the ground state of the
system is by propagation in imaginary time. Consider the Schr\"odinger equation
\begin{equation}
\imath \hbar \frac{\partial \ket{\psi(t)}}{\partial t} = \hat{H} \ket{\psi(t)}.
\end{equation}
The transformation $ \tau = \imath t$ lead to the equation
\begin{equation}
\hbar \frac{\partial \ket{\psi(\tau)}}{\partial \tau} = - \hat{H} \ket{\psi(\tau)}.
\end{equation}
The formal solution for this equation, with initial condition $\ket{\psi(0)} = \ket{\psi_0}$ is $\ket{\psi(\tau)} = \exp\left( -\frac{\tau}{\hbar} H \right) \ket{\psi_0}$. The initial state $\ket{\psi_0}$ can be written as a linear combination of Hamiltonian's eingenvectors:
\begin{equation}
\ket{\psi_0} = \sum_i c_i \ket{\phi_i},
\end{equation}
where $\hat{H} \ket{\phi_i} = E_i \phi_i$ for $i = 0,1,2,\ldots$~. In this basis $\ket{\psi(\tau)}$ can be written as follow
\begin{equation}
\ket{\psi(\tau)} = \sum_i c_i \exp\left( -\frac{\tau}{\hbar} E_i \right) \ket{\phi_i}.
\end{equation}
Taking $E_0$ as the ground--state energy, we can rearrange the previous equation
\begin{equation} \label{eq:imag-evolution}
\ket{\psi(\tau)} = \exp \left( -\frac{\tau}{\hbar} E_0 \right) \sum_i c_i  \exp\left( -\frac{\tau}{\hbar} \Delta E_i \right) \ket{\phi_i}.
\end{equation}
where $\Delta E_i = E_i - E_0 > 0,  \forall i > 0$. We now take the limit for $\tau \rightarrow +\infty$. As long as the initial state is not orthogonal to the ground state, namely $c_0 \neq 0$, the leading term in the sum of Eq.~\eqref{eq:imag-evolution} is given by the ground state
\begin{equation}
\lim_{\tau \rightarrow +\infty} \ket{\psi(\tau)} = \exp \left( -\frac{\tau}{\hbar} E_0 \right) c_0 \ket{\phi_0}.
\end{equation}
Thus, evolving the initial state for a sufficient amount of time, let us to reach an  approximation of the ground state.

We implement the evolution operator in imaginary time using the same Trotter--Suzuki decomposition and Hamiltonian splitting as for the real time evolution. The operator reads as
\begin{equation}
\hat{U}(\tau) =   \exp\left(-\frac{\tau}{\hbar} \left( -\frac{\hbar^2}{2 m \Delta^2} \left[ \hat{A}_x + \hat{B}_x + \hat{A}_y + \hat{B}_y - 4 \hat{I} \right] + \hat{V} \right) \right)
\end{equation} 
so in the first Trotter--Suzuki approximation we have
\begin{align}
\hat{U}_1(t) = \exp\left(-\frac{\tau}{\hbar}\left(\hat{V} + \frac{2 \hbar^2}{m \Delta^2} \hat{I}\right) \right) & \exp\left(\alpha_\tau \hat{A}_x \right) \exp\left( \alpha_\tau \hat{B}_x \right) \cdot \\ \cdot & \exp\left( \alpha_\tau \hat{A}_y \right) \exp\left( \alpha_\tau \hat{B}_y \right), \nonumber
\end{align}
where $\alpha_\tau = \frac{\hbar \tau}{2m\Delta^2}$. Using the equality
\begin{equation}
\exp( \alpha_\tau \sigma) = I \cosh(\alpha_\tau) +  \sigma \sinh(\alpha_\tau)
\end{equation}
we can calculate the exponential of $A$ and $B$ as
\begin{equation}
\exp\left( \alpha_\tau A_x \right) = I_x \cosh(\alpha_\tau) + \sinh(\alpha_\tau) A_x
\end{equation}
\begin{equation}
\exp\left( \alpha_\tau B_x \right) = I_x (\cosh(\alpha_\tau)(1-\delta_{x,0}) + \delta_{x,0}) +  \sinh(\alpha_\tau) B_x
\end{equation}
Finally, in the second order approximation we have:
\begin{equation}
\hat{U}_2(\tau) = \hat{U}_1\left( \frac{\tau}{2} \right)^\mathrm{T} \hat{U}_1\left(\frac{\tau}{2}\right)
\end{equation}

