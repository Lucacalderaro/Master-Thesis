\chapter{Introduction}
The description of the natural processes and the phenomena that happen around us comes with the definition of a model. In physics, a model is a mathematical framework that tries to reproduce what we observe as much as possible. Yet, very often the predictions of a model come with the solution of equations -- which may be a rather challenging problem itself. A typical approach is to define the energy of the system, which contains the interactions between the components of the system and the kinetic of the particles -- when these are allowed to move --. This in turn leads to the action of the system and the equations of motion by means of the least action principle. At this stage, many features of the system can be already found without the need to solve the equations of motion. 
%For instance, examining the symmetries of the system, one can find the quantities that do not change in time during evolution, or (many-body system)
But when we are interested in the details of the dynamic of the system we have to solve them.

The dynamic of many interesting systems is described by equations difficult to solve analytically. The solution of the Cauchy problem, that is to find for each time the state that solve the equation, given an initial state,