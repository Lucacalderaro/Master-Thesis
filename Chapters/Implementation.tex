\chapter{Mathematical Implementation}
-schrodinger equation
-mathematical explanation
-program implement an hamiltonian 
-discretization of the space
-formula

-problem and geometry considered
-good trotter approx based on suitable hamiltonian decomposition

When it comes to solve differential equations that govern the behavior of a physical system, a good approximation is often needed to find an explicit solution. As we have seen in the previous chapter, the Trotter-Suzuki approximation is a very good choise for this purpose. 

%%%%%%%%%%%%%%%%%%%%%%%%%%%%%%%%%%%%%%%%%%%%%%%%%%%%%%%%%%%%%%%%%%%%%%%%
\section{Hamiltonian decomposition}
We consider the following hamiltonian operator
\begin{equation} \label{eq:hamiltonian-implementation}
\hat{H} = \frac{\hat{P}_x^2 + \hat{P}_y^2}{2m} + \hat{V},
\end{equation}
where $m$ is the particle mass and $\hat{V}$ is the external potential.
Using the coordinate reppresentation of the operators, for the total momentum we have
\begin{subequations}
\begin{align}
\bra{x,y} \left( P_x^2 + P_y^2 \right) \ket{\psi} = & \int \mathrm{d}x' \mathrm{d}y' \bra{x,y} \left( P_x^2 + P_y^2 \right) \ket{x',y'} \braket{x',y'|\psi} \label{eq:tot-momentum1} \\
= & -\hbar^2 \left( \nabla_x^2 + \nabla_y^2 \right) \psi(x,y) \label{eq:tot_momentum}
\end{align}
\end{subequations}
We consider the discretization of the continuum space into a uniform mesh, where $\Delta$ is the distance between any two consecutive points. We use the couple $(i,j)$ to label the points of the mesh, with $i,j = 1,\ldots , N$, so that $\psi(x,y) \rightarrow \psi_{i,j}$ and $\ket{x,y} \rightarrow \ket{i,j}$. Using as second derivative approximation the  second-order derivative central difference:
\begin{equation}
\frac{\partial^2 \psi}{\partial x^2} \bigg|_{i,j} = \frac{\psi(i+1,j) - 2 \psi(i,j) + \psi(i-1,j)}{\Delta^2} + O(\Delta^2)
\end{equation}
the Eq.~\eqref{eq:tot_momentum} becomes
\begin{equation} \label{eq:discr-tot-momentum}
\bra{i,j} \left( P_x^2 + P_y^2 \right) \ket{\psi} = -\frac{\hbar^2}{\Delta^2} \left( \psi(i+1,j) + \psi(i,j+1) + \psi(i-1,j) + \psi(i,j-1) - 4 \psi(i,j) \right).
\end{equation}
To make the matrix elements of the total momentum explicit, let us rewrite the Eq.~\eqref{eq:discr-tot-momentum} using kroneker's delta
\begin{equation} \label{eq:delta-k-momentum}
\bra{i,j} \left( P_x^2 + P_y^2 \right) \ket{\psi} = \sum_{k,l} -\frac{\hbar^2}{\Delta^2} \left[ \left( \delta_{i+1,k} + \delta_{i-1,k} \right) \delta_{j,l} + \left( \delta_{j+1,l} + \delta_{j-1,l} \right) \delta_{i,k} - 4 \delta_{i,k} \delta_{j,l} \right] \psi_{k,l}.
\end{equation}
Since the discretization of Eq.~\eqref{eq:tot-momentum1} led to the following equation
\begin{subequations}
\begin{align}
\bra{i,j} \left( P_x^2 + P_y^2 \right) \ket{\psi} = & \sum_{k,l} \bra{i,j} \left( P_x^2 + P_y^2 \right) \ket{k,l} \braket{k,l|\psi} \\
= & \sum_{k,l} \bra{i,j} \left( P_x^2 + P_y^2 \right) \ket{k,l} \psi_{k,l} \label{eq:discr-momentum}
\end{align}
\end{subequations}
comparing Eq.~\eqref{eq:delta-k-momentum} and Eq.~\eqref{eq:discr-momentum}, the matrix element of the total momentum is
\begin{equation} \label{eq:delta-kroneker}
\bra{i,j} \left( P_x^2 + P_y^2 \right) \ket{k,l} = -\frac{\hbar^2}{\Delta^2} \left[ \left( \delta_{i+1,k} + \delta_{i-1,k} \right) \delta_{j,l} + \left( \delta_{j+1,l} + \delta_{j-1,l} \right) \delta_{i,k} - 4 \delta_{i,k} \delta_{j,l} \right].
\end{equation}

Now, we introduce two operators that will let us split the hamiltonian into a sum of operators easy to exponentiate. We define:
\begin{equation}
A_{i,k} = \begin{cases} \delta_{i+1,k}, & \mbox{if } k\mbox{ is odd} \\ \delta_{i-1,k}, & \mbox{if } k\mbox{ is even} \end{cases}
\end{equation}
and
\begin{equation} 
B_{j,l} = \begin{cases} \delta_{j-1,l}, & \mbox{if } l\mbox{ is odd} \\ \delta_{j+1,l}, & \mbox{if } l\mbox{ is even} \end{cases}
\end{equation}
Seen as matrices, these operators are block diagonal matrices and can be written as:
\begin{equation}
A = \begin{pmatrix}
0 & 1 \\
1 & 0 \\
 & & 0 & 1 \\
 & & 1 & 0 \\
 & & & & 0 & 1 \\
 & & & & 1 & 0 \\
 & & & &  & & \ddots \\
\end{pmatrix}
\quad B = \begin{pmatrix}
0 \\
& 0 & 1 \\
& 1 & 0 \\
& & & 0 & 1 \\
& & & 1 & 0 \\
& & & & & 0 & 1 \\
& & & & & 1 & 0 \\
& & & & &  & & \ddots \\
\end{pmatrix}
\end{equation}
Using the new operators we can rewrite Eq.~\eqref{eq:delta-kroneker} as follow
\begin{equation}
\bra{i,j} \left( P_x^2 + P_y^2 \right) \ket{k,l} = -\frac{\hbar^2}{\Delta^2} \left[ \left( A_{i,k} + B_{i,k} \right) \delta_{j,l} + \left( A_{j,l} + B_{j,l} \right) \delta_{i,k} - 4 \delta_{i,k} \delta_{j,l} \right]
\end{equation}
Using the operator notation we have
\begin{equation} \label{eq:mom-decomposition}
\hat{P}_x^2 + \hat{P}_y^2 = -\frac{\hbar^2}{\Delta^2} \left[ \hat{A}_x + \hat{B}_x + \hat{A}_y + \hat{B}_y - 4 \hat{I} \right],
\end{equation}
where the label indicates the axes to which the operator acts, so that the following commutation rules are satisfied:
\begin{equation}
[\hat{A}_x, \hat{A}_y] = 0 \quad [\hat{A}_x, \hat{B}_y] = 0 \quad [\hat{B}_x, \hat{A}_y] = 0 \quad [\hat{B}_x, \hat{B}_y] = 0.
\end{equation} 
By mean of the Eq.~\eqref{eq:mom-decomposition}, we can write the hamiltonian~\eqref{eq:hamiltonian-implementation} as
\begin{equation} \label{eq:hamiltonian-decomposition}
\hat{H} = -\frac{\hbar^2}{2 m \Delta^2} \left[ \hat{A}_x + \hat{B}_x + \hat{A}_y + \hat{B}_y - 4 \hat{I} \right] + \hat{V}
\end{equation}

%%%%%%%%%%%%%%%%%%%%%%%%%%%%%%%%%%%%%%%%%%%%%%%%%%%%%%%%%%%%%%%%%%%%%%%%%%%%%%%%%%%%%%%%%
\section{Evolution operator}
In the previous section we shown how to split the Hamiltonian in the discrete space approximation. Now we explicitly calculate the Trotter-Suzuki decomposition for the evolution operator. Using the hamiltonian decomposition~\eqref{eq:hamiltonian-decomposition}, the evolution operator $\hat{U}(t) = \exp(-\frac{\imath t}{\hbar} H)$ can be written as follow 
\begin{align}
U(t) = &  \exp\left(-\frac{\imath t}{\hbar} \left( -\frac{\hbar^2}{2 m \Delta^2} \left[ \hat{A}_x + \hat{B}_x + \hat{A}_y + \hat{B}_y - 4 \hat{I} \right] + \hat{V} \right) \right) \\
= & \exp\left(-\frac{\imath t}{\hbar}\left(V + \frac{2 \hbar^2}{m \Delta^2} I\right) \right) \exp\left(\imath \alpha A_x \right) \exp\left(\imath \alpha B_x \right) \exp\left(\imath \alpha A_y \right) \exp\left(\imath \alpha B_y \right) + O(t^2)
\end{align}
where we defined $\alpha = \frac{\hbar t}{2m\Delta^2}$. 

Using the equality 
\begin{equation}
\exp(\imath \alpha \sigma) = I \cos(\alpha) + \imath \sigma \sin(\alpha)
\end{equation}
it is straightforward to calculate the exponential of the operators $A$ and $B$, since they are diagonal matrices of pauli matrix. Then, we get
\begin{equation}
\exp\left(\imath \alpha A_x \right) = I_x \cos(\alpha) + \imath \sin(\alpha) A_x
\end{equation}
\begin{equation}
\exp\left(\imath \alpha B_x \right) = I_x (\cos(\alpha)(1-\delta_{x,0}) + \delta_{x,0}) + \imath \sin(\alpha) B_x
\end{equation}
So using

%%%%%%%%%%%%%%%%%%%%%%%%%%%%%%%%%%%%%%%%%%%%%%%%%%%%%%%%%%%%%%%%%%%%%%%%%%%%%%%%%%%%%%%%%
\section{Evolution towards the ground-state}
A reliable and easily-implemented method of obtaining the ground state of the
system is by propagation in imaginary time. Consider the schr\"odinger equation
\begin{equation}
\imath \hbar \frac{\partial \ket{\psi}}{\partial t} = H \ket{\psi},
\end{equation}
The transformation $ \tau = \imath t$ led to the equation
\begin{equation}
\hbar \frac{\partial \ket{\psi}}{\partial \tau} = - H \ket{\psi}.
\end{equation}
The formal solution for this equation is $\ket{\psi(\tau)} = \exp\left( -\frac{\tau}{\hbar} H \right) \ket{\psi_0}$. $\ket{\psi_0}$ can be expanded in the hamiltonian eingenvalue basis
\begin{equation}
\ket{\psi_0} = \sum_i c_i \ket{\phi_i},
\end{equation}
where $H \ket{\phi_i} = E_i \phi_i$ for $i = 0,1,2,\ldots$. In this basis $\ket{\psi(\tau)}$ can be written as follow
\begin{equation}
\ket{\psi(\tau)} = \sum_i c_i \exp\left( -\frac{\tau}{\hbar} E_i \right) \ket{\phi_i}.
\end{equation}
Taking $E_0$ as the ground-state energy, we can rearrange the previous equation
\begin{equation} \label{eq:imag-evolution}
\ket{\psi(\tau)} = \exp \left( -\frac{\tau}{\hbar} E_0 \right) \sum_i c_i  \exp\left( -\frac{\tau}{\hbar} \Delta E_i \right) \ket{\phi_i}.
\end{equation}
where $\Delta E_i = E_i - E_0 > 0,  \forall i > 0$. We now take the limit for $\tau \rightarrow +\infty$. As long as the initial state is not orthogonal to the ground state, namely $c_0 \neq 0$, the leading term in the sum of Eq.~\eqref{eq:imag-evolution} is given by the ground state
\begin{equation}
\lim_{\tau \rightarrow +\infty} \ket{\psi(\tau)} = \exp \left( -\frac{\tau}{\hbar} E_0 \right) c_0 \ket{\phi_0}.
\end{equation}
Practically, the simulation evolves the state for a sufficient amount of time, to let the states different from the ground state to decay. However, the codomain of the wave-function must be included in the range of value that can be stored in the memory. For this reason, we need to perform a normalization of the wave function. At that point, if we need a better approximation, we repeat the process evolving the state and normalizing it.


The code implements the evolution operator in immaginary time, using the same Trotter-Suzuki decomposition, and hamiltonian splitting as for the real time evolution. The operator reads
\begin{equation}
U(\tau) =   \exp\left(-\frac{\tau}{\hbar} \left( -\frac{\hbar^2}{2 m \Delta^2} \left[ \hat{A}_x + \hat{B}_x + \hat{A}_y + \hat{B}_y - 4 \hat{I} \right] + \hat{V} \right) \right)
\end{equation} 
so in the first Trotter-Suzuki approximation
\begin{equation}
U(t) = \exp\left(-\frac{\tau}{\hbar}\left(V + \frac{2 \hbar^2}{m \Delta^2} I\right) \right) \exp\left(\alpha_\tau A_x \right) \exp\left( \alpha_\tau B_x \right) \exp\left( \alpha_\tau A_y \right) \exp\left( \alpha_\tau B_y \right) + O(\tau^2),
\end{equation}
where $\alpha_\tau = \frac{\hbar \tau}{2m\Delta^2}$. Now we use the equality
\begin{equation}
\exp( \alpha_\tau \sigma) = I \cosh(\alpha_\tau) +  \sigma \sinh(\alpha_\tau)
\end{equation}
throught which we can calculate the exponential of $A$ and $B$
\begin{equation}
\exp\left( \alpha_\tau A_x \right) = I_x \cosh(\alpha_\tau) + \sinh(\alpha_\tau) A_x
\end{equation}
\begin{equation}
\exp\left( \alpha_\tau B_x \right) = I_x (\cosh(\alpha_\tau)(1-\delta_{x,0}) + \delta_{x,0}) +  \sinh(\alpha_\tau) B_x
\end{equation}
